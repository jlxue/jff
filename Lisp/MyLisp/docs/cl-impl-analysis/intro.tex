\startcomponent intro
\product cl-impl-analysis

\chapter{开源 CL 实现介绍}

\section{Parrot CL}

Parrot 上的 CL 实现,现在已不大维护:

svn co https://svn.parrot.org/languages/lisp/trunk lisp

\section{Kea-CL}

Parrot 上的另一个 CL 实现,基于 CMUCL/SBCL 的构造原理,现在已
不大维护:

svn co https://rgrjr.dyndns.org/svn/kea-cl/trunk kea-cl

\section{CMU CL}

高效的 CL 实现,对 i18n 以及 MS Windows 支持不好:

cvs -d :pserver:anonymous@common-lisp.net:\backslash\crlf
/project/cmucl/cvsroot co src

\section{SBCL}

派生自 CMU CL:

cvs -d :pserver:anonymous@sbcl.cvs.sourceforge.net:\backslash\crlf
/cvsroot/sbcl co sbcl

\section{Clozure CL}

从商业 CL 实现 MCL 派生出来的开源版本,以前叫 OpenMCL,在 Clozure CL
1.2 发布时由于听说商业 MCL 也要开源,为了避免 Open MCL 和 OpenMCL 让人
混淆,因此改名叫 Clozure CCL 了:

svn co http://svn.clozure.com/publicsvn/openmcl/\backslash\crlf
trunk/linuxx86/ccl

\section{CLisp}

cvs -d :pserver:anonymous@clisp.cvs.sourceforge.net:\backslash\crlf
/cvsroot/clisp co clisp

\section{ECL}

cvs -d :pserver:anonymous@ecls.cvs.sourceforge.net:\backslash\crlf
/cvsroot/ecls co ecl

\section{ABCL}

JVM 上的 Common Lisp 实现:

svn co svn://common-lisp.net/project/armedbear\backslash\crlf
/svn/trunk/abcl

\section{GCL}

已长时间没有维护,自带一套 ANSI Common Lisp 测试代码。

\section{scmtoy-llvm}

chylli@newsmth 实现的 Scheme -> LLVM Assembly 编译器和解释器,这个
不是 Common Lisp 实现,但对于理解 Lisp 实现的构造过程不无裨益:

git clone git://github.com/chylli/scmtoy-llvm.git

\stopcomponent

