\startcomponent workflow
\product git-way-explained

\section{工作流程}

由于 GIT 太灵活了太繁复了,所以 GIT 用户常常结合自己的使用经历
总结 GIT 的工作流程,在 GIT 源码的 Documentation/howto 和 GIT
邮件列表中有很多这样的总结。

我个人的体会有如下几点:

\startitemize[n,packed,broad]
\item 不要往不是自己手动建立的分支上直接提交。

这里说的主要是 tracking branch,也就是 .git/config 中 [branch] 节
配置的那些分支,因为 remote branch 一般都不会误提交上去。我感觉
压根就不应该使用 tracking branch,因为 git pull 的合并百分之九十
的情况下都不是想要的结果,而且经常发生冲突,即使确实想合并,执行
下 \type{git merge} 也不费事,不建议使用 pull 命令。另外我常常希望上游
分支的历史尽量是线性的以方便查看,所以我常用 rebase 和 cherry-pick
而不是 merge 或者 pull 来处理工作分支。

我跟踪别人分支的办法是用 \type{git fetch repos src:dst} 或者 \type{git remote update},
前者比较随意,目标分支可以不在 refs/remotes/ 这个命名空间里,后者比较正规。

有了 remote branch 后我再建立个 normal branch 就行了
\footnote{git checkout -b my origin/master 会自动在 .git/config 中
把 my 分支配置为 tracking branch,去掉这个配置就行了。},想提交到正式库时,
先更新 remote branch,然后 rebase normal branch,然后 push 或者 fetch 到
正式库中的上游分支。

这样最少可以只牵涉到三个分支:正式库里的上游分支,本地库里的
remote branch 和工作分支。而如果要用 tracking branch 的话,为了
避免 pull 时冲突,我不会把 tracking branch 作为工作分支,这样
tracking branch 和 remote branch 的功能就完全重复了。实际上
我很少用 pull,所以 tracking branch 对我没用,只会让人迷糊。

如果我的本地库要提供一个分支让人跟踪,那么我一般是再建立个工作分支,
可以随意提交修改,改好后再建立个分支用\type{git merge --squash} 和
\type{git cherry-pick}来整理这些修改,弄完后\type{git rebase pub}
然后用 \type{git push . my:pub} 更新我的公开分支,不往公开分支上直接提交。

\item 不要往 HEAD 指向的分支 push。

push 的目标分支应该不会被 checkout 出来,因为 push 过后,head 指向的
位置变了,导致 index 跟 HEAD 版本不匹配而出现很莫名的文件被修改状态,
虽然可以用 \type{git reset --hard} 或者更细致的
\type{git reset && git checkout paths...} 纠正,但总是不便。

\item 尽量不要回滚公开分支。

公开分支如果回滚了,比如被 rebase,commit --amend 或者强制修改了 head 的指向,
都会给别人从这个分支上 pull 造成麻烦,即使执行 pull 的人本地没有做
任何修改也会出现很莫名的冲突。
\stopitemize

\stopcomponent

