\startcomponent terminology
\product git-way-explained

\section{GIT 里的术语定义}

GIT 里用 blob, tree, commit, tag 四种存储对象来描述版本控制
所需信息,每一种存储对象都是以类型名字、内容长度以及内容计算出
的 SHA1 摘要值命名,存放在 .git/objects/ 目录下的文件。

\definedescription[term][location=top,headstyle=bold,width=broad]
\startterm{blob}
表示\in{第}{节}[basis]中的文件, blob 只有文件数据和长度,不保存名字、
文件类型、权限等信息,恰如其名;
\stopterm

\startterm{tree}
表示\in{第}{节}[basis]中的目录,不包括像 Subversion 那样的 property
\footnote{GIT 有 git check-attr 和 git notes 命令,但这些信息并不是跟
Subversion 一样记录在版本历史里,而是单独存放的},只有文件权限位(无属组
和属主信息)、对象类型(blob 还是tree)、SHA1 摘要值、文件或子目录名字。早
期的 git 实现中tree 对象保存的是整个文件树的路径名清单,现在的实现中一个
tree 对象只保存某个目录下的一层信息;
\stopterm

\startterm{commit}
表示\in{第}{节}[basis]中的版本;
\stopterm

\startterm{tag}
表示\in{第}{节}[basis]中静态标签。
\stopterm

值得一提的是 GIT 自称 content tracker 而非版本控制工具。这一思想在 blob
和 tree 记录的信息上有所体现,blob 和 tree只记录了绝对必要的{\em 数据}(
权限位虽然不必要但是在 UNIX 类系统上非常有用),因为是 content tracker,
所以 blob 和 tree不关心作者名字、类似 Subversion 那样的额外属性,这样两
个作者声称数据内容一致(额外属性可以不一致)时,必然可以共用 blob和 tree
存储对象。这样做的好处是存储模型简单,节约空间,有利于分布式使用,坏处是
少了 Subversion 那样方便的跟目录、文件关联的纳入版本控制的属性功能。

content tracker 这个名字还体现在对 file 和 blob 的区分对待
上:file 指 blob + file name。git 所关心的是文件内容的流向,
并不显式记录复制、移动这些信息,这些信息是运行时根据文件
内容流向动态分析出来的。这样做的好处是减轻了使用者的负担,
据个人所见,很多 Subversion 用户在应该用 svn copy 时却用 svn add,
导致版本库迅速膨胀,用 svn log 也看不出来复制、移动信息。
GIT 这样做的坏处自然是动态分析需要时间,单次变更集(change set)
太大时效果不行,这就寄希望于更好的探测算法了──起码我们将来
可以很容易补救。

file 和 blob 的区别体现在文件历史上,在 git 里,一个文件
的历史是这个文件路径曾经指代的文件内容变化,也就是一个
路径所牵涉到的历史,而在 Subversion 里一个文件历史指的是
这个文件在版本变迁中内容的改变、路径的改变历史,因此如下
操作在 git log 和 svn log 结果迥然不同:

\starttyping
    git/svn add a && commit && git/svn mv a b && commit
    git log -- a 会显示两次提交(一次新增,一次删除);
    git log -- b 会显示一次提交;
    svn log a 会显示 a 没有纳入版本控制;
    svn log b 会显示两次提交。
\stoptyping

上面两个命令虽然形式类似,但语义是相当不同的,svn
是在当前版本的文件树中找文件,找到后回溯它的历史,
包括内容变化和名字变化,而 git 是从当前版本回溯历史,
过滤出修改中涉及指定路径的修改(svn 和 git 各有选项能模拟彼此的效果)。

关于 file 和 blob 的区别以及 GIT 不显式记录移动、复制操作
的设计,早期在 GIT 的邮件列表上引起极大的争议,为此爆发了
GIT 之父 Linus Torvalds 惊天地泣鬼神前无古人后无来者先天下
之忧而忧后天下之乐而乐的“I'm right”宣言\footnote{\from[linus-word]},
以及现任 GIT 维护者 Junio C Hamano 同样相当精彩的“read it now”
警言\footnote{\from[junio-word]}。

虽然 GIT 是 stupid content tracker,可实际上我们是拿它
当 version control system 来使。GIT 包含两种命令,
底层命令和高层命令,底层命令实现了 content tracker,高层
命令封装底层命令模拟出了 version control system──
其实语义跟通常的版本控制系统还是有差别的,这往往让人迷惑。


在 GIT 中比较容易混淆的一个术语是引用(reference, 简称 ref)类型,
从字面上理解,引用就是对上面四种对象的标记,这是 GIT 对动态标签、
静态标签等特性的实现机制,虽然是实现机制,但对使用有些晦涩的影响。

GIT 中的引用的主要用途:
\startterm{head}
head 就是动态标签,分支头,记录一个分支的最新版本。保存在 .git/refs/heads/
目录下,可以用全名 refs/heads/xxx 也可以用简名xxx。

这种 ref 被 git branch、git checkout -b、git update-ref 创建、修改。
\stopterm

\startterm{tag}
tag 就是静态标签,但从 GIT {\em 实现}里分两种,一种实现为普通的 tag object,保存
在 .git/objects 下,另外在 .git/refs/tags/ 下额外保存一个 tag ref,内容是 tag 对
象的 SHA1 摘要值。 这种 tag 用 git tag -a、git tag -u 或者 git tag -s 命令创建
(注意如果没提供 -a/-u/-s 但提供了 -m 则隐含了 -a),这种 tag 可以包含日志、签名、
时间、创建者信息。

另一种是用\type{git tag TAGNAME COMMIT-ISH} 创建的(没有 -m/-a/-u/-s),这时没
有创建 tag object,只是在 .git/refs/tags/ 下创建了 tag ref。这种 tag 被
称为轻量级 tag,只有一个名字跟一个 SHA1 值,用于命名对象以备忘。

两种 tag 实现方式都可以用 git tag 命令查看、修改。 另外 git update-ref 也能创建、
修改 reference,也就是 .git/refs/ 下的文件。
\stopterm

除了上面两种主要用途外,ref 还被用于保存 note (被 git notes 命令使用) 以
及 remote tracking branch(被 git remote、git fetch、git push、git pull
等使用)。

引用存放在 .git/refs/heads 和 .git/refs/tags 下(引用可以
被打包集中存放在 .git/info/refs 或 .git/packed-refs 文件
中)。

[ XXX: 术语跟 CVS、Subversion 很不一样,可怜我等呆瓜脑袋]


更混乱的是还有一个 symbolic ref,用 git symbolic-ref 命令维护,
保存在 .git/ 目录下的符号链接或者内容为 refs: refs/.... 的普通文件,
比如 .git/HEAD 内容一开始是 refs: refs/heads/master,表示{\em 当前}
head 是master。

需要注意的是在命令行引用分支时,如果 symbolic ref 跟
ref 重名,那么引用后者时要用 refs/heads/xxx 或者 heads/xxx
这样的名字,关于这点在 git help rev-parse 的 SPECIFYING REVISIONS
一节有说明。

[ XXX: 够混淆]


下面从实例看看它们各自的存储方式。
\starttyping
$ mkdir t && cd t
$ git init          # 初始化版本库,默认放在工作目录里
$ echo hello > a
$ git add a
$ git commit -m "first"
$ git tag v0
$ ls .git/objects/??/*
.git/objects/09/76950c1fdbcb52435a433913017bf044b3a58f
.git/objects/51/ca0ad1685f36558c01ec400350d988f44176bd
.git/objects/ce/013625030ba8dba906f756967f9e9ca394464a
$ git cat-file -t 097695
tree
$ git cat-file -t 51ca0a
commit
$ git cat-file -t ce0136
blob
$ git cat-file -p 09769a
100644 blob ce013625030ba8dba906f756967f9e9ca394464a    a
$ git cat-file -p 51ca0a
tree 0976950c1fdbcb52435a433913017bf044b3a58f
author Liu Yubao <yubao.liu@gmail.com> 1224781865 +0800
committer Liu Yubao <yubao.liu@gmail.com> 1224781865 +0800

first
$ git cat-file -p ce0136
hello
$ find .git/refs/ -type f | while read f ; do echo $f; cat $f; done
.git/refs/tags/v0
51ca0ad1685f36558c01ec400350d988f44176bd
.git/refs/heads/master
51ca0ad1685f36558c01ec400350d988f44176bd
\stoptyping

从上面可以看到:我们创建了一个 blob (git add 创建)、
一个 tree 和 一个 commit(git commit 创建)、一个分支
master (git commit 第一次提交时创建)和一个轻量级 tag
v0 (git tag 创建),它们的关系为:

\starttyping
master 和 v0 -> commit(51ca0a) -> tree(097695) -> blob(ce0136)
\stoptyping

\starttyping
$ git tag -m "this is v1" v1
\stoptyping
这会新增 .git/objects/83/50f23b202c16803a7b20d2a8e37015d674babc
和 .git/refs/tags/v1。

\starttyping
$ git cat-file -p 8350f2
object 51ca0ad1685f36558c01ec400350d988f44176bd
type commit
tag v1
tagger Liu Yubao <yubao.liu@gmail.com> Fri Oct 24 01:31:01 2008 +0800

this is v1
$ cat .git/refs/tags/v1
8350f23b202c16803a7b20d2a8e37015d674babc
\stoptyping

可以看到新增了一个 tag ref v1 指向 tag object 8350f2,后者指向
commit object 51ca0a。

tag 可以指向四种 object,常用的是指向 tree 和 commit。

[ XXX: tag object 本身并没有记录入 commit 或者 tree 中,而 tag object
内容包含了名字,那么 tag ref 只是为了加速访问而存在的东西了。tag 是为了
助记,那么 tag object 的存在就滑稽了,其 SHA1 摘要用户一般记不住,tag
object 的内容也完全可以放入 tag ref 中,即使签名也能用 ASCII 形式的
签名附在 tag ref 文件内容末尾。]

从上面的 commit 对象内容可以看出 commit 区分了 author 和 committer,
这对于协同开发是很方便的。

关于分支很重要的一点就是分支其实是用 head 也就是一个
浮动标签表示的,我们可以很暴力的直接修改甚至删除重建
这个标签,这样就能改变一个分支名指代的分支了,这招
可以说是被 GIT 的 pull/push/fetch/merge/rebase/reset/revert
折腾的头晕目眩之余的必杀技:
\starttyping
    git push -f . test:master
    git fetch -f . test:master
    git branch -D master && git branch master test
    git update-ref refs/heads/master test
    cp .git/refs/heads/test .git/refs/heads/master
\stoptyping
然后执行 git reset --hard,就能恢复 master 这个 head 让它
与 test 这个 head 指向同一个版本。

\stopcomponent

